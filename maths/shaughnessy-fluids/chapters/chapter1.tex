Problems 1.1 - 1.15 are discussion questions.

%    ----    ----    Q 1.16    ----    ----
\begin{question}
  Q 1.16. Consider steel and aluminium sheets of $6\si{mm}$ thickness; each metal sheet is subjected to a shear stress of $2 \times 10^{5} \si{psi}$. Compare the magnitudes of the displacements.
\end{question}

\begin{solution}
  We know that for solids the following relationship holds:
  \begin{equation}\label{eq:shear-solids}
    \tau = G \gamma
  \end{equation}
  where $\tau$ is the shear stress, $G$ is the shear modulus, and $\gamma = \Delta x / \Delta y$, the shear strain.
  \\

  \noindent \underline{Steel}
  \begin{itemize}
    \item shear modulus, $G = 11.3 \times 10^6 \si{psi}$ for cast steel \parencite{engineering03shearmodulus}.
    \item magnitude of shear stress, $\tau = 2 \times 10^5 \si{psi}$
    \item height of sample, $ \Delta y = 6 \times 10^{-3} \si{psi}$
  \end{itemize}

  Then, we have

  \begin{align*}
    2 \times 10^5 \si{psi} &= \frac{ 11.3 \times 10^6 \si{psi} ~ \Delta x }{ 6 \times 10^{-3} \si{m} } \\
    \Delta x &= \frac{ 2 \times 10^{5} \si{psi} \times 6 \x 10^{-3} \si{m} }{ 11.3 \times 10^{6} \si{psi} } \\
    &= 1.07 \x 10^{-4} \si{m} \text{ or } 0.11 \si{mm}
  \end{align*}

  \noindent \underline{Aluminium}
  \begin{itemize}
    \item shear modulus, $G = 3.9 \times 10^6 \si{psi}$ for aluminium alloys \parencite{engineering03shearmodulus}.
    \item magnitude of shear stress, $\tau = 2 \times 10^5 \si{psi}$
    \item height of sample, $ \Delta y = 6 \times 10^{-3} \si{psi}$
  \end{itemize}

  Then, we have

  \begin{align*}
    2 \times 10^5 \si{psi} &= \frac{ 3.9 \times 10^6 \si{psi} ~ \Delta x }{ 6 \times 10^{-3} \si{m} } \\
    \Delta x &= \frac{ 2 \times 10^{5} \si{psi} \times 6 \x 10^{-3} \si{m} }{ 3.9 \times 10^{6} \si{psi} } \\
    &= 3.08 \x 10^{-4} \si{m} \text{ or } 0.31 \si{mm}
  \end{align*}

  So we have a displacement that is around three times larger for aluminium than for steel for the same shear stress applied, this is consistent with our intuition that aluminium is much more malleable than steel.
\end{solution}

%    ----    ----    Q 1.17    ----    ----

\begin{question}
  Q 1.17 Consider steel and aluminium sheets of $5\si{mm}$ thickness. If the steel sheet is subjected to a shear stress of $8 \x 10^5 \si{psi}$, calculate the magnitude of the shear stress that must be applied to the aluminium sheet so that both materials experience the same displacment in the direction of the applied force.
\end{question}

\begin{solution}
  Using \eqref{eq:shear-solids},

  \begin{align*}
    \frac{ \tau_{al} \Delta y }{ G_{al} } = \Delta x_{al} = \Delta x_{steel} = \frac{ \tau_{steel} \Delta y }{ G_{steel} }
  \end{align*}

  since we need the displacement of both sheets to be the same, and they have the same thickness, $\Delta y$. We now just need to rearrange for $\tau_{al}$ using the data form question 1.16:

  \begin{align*}
    \tau_{al} &= \frac{ \tau_{steel} G_{al} }{ G_{steel} } \\
    &= \frac{ (8 \x 10^5 \si{psi})(3.9 \times 10^{6} \si{psi}) }{ 11.3 \times 10^{6} \si{psi} }\\
    & = 2.76 \x 10^5 \si{psi}.
  \end{align*}
\end{solution}

%    ----    ----    Q 1.18    ----    ----

\begin{question}
  Q 1.18 A sample of motor oil has been tested in a flat plate shearing device with the following results: $\tau = 7.7 \si{lb_{f}/ft^2}$ for a plate separation distance of $0.25 \si{in}$ and a top plate velocity of $25 \si{ft/s}$. Determine the viscosity of the fluid.
\end{question}

\begin{solution}
  Here we apply Newton's Law of Viscosity
  \begin{equation}\label{eq:newton-viscosity}
    \tau = \mu \frac{ d \gamma }{ dt }
  \end{equation}
  where $\mu$ is the dynamic viscosity of the fluid, $\gamma$ is the shear strain and $u$ is the fluid speed in the direction of the shear stress. We have:

  \begin{itemize}
    \item $\tau = 7.7 \si{lb_{f}/ft^2}$
    \item plate separation $\Delta y = 0.25 \si{in}$
    \item top plate speed in the direction of the shear stress after the stress is applied $u = 25 \si{ft/s}$
  \end{itemize}

  Note that we also know that
  \begin{equation}
    \tau = \mu \frac{ du }{ dy }.
  \end{equation}
  Hence we can write
  \begin{equation}
    \tau = \mu \frac{ \Delta u }{\Delta y}
  \end{equation}
  where $\Delta u = 25 \si{ft/s}$, since before the shear stress is applied the top plate velocity is $0$, and $\Delta y = 0.25 \si{in} = 0.25 \times 1/12 \si{ft} = 1/48 \si{ft}$, which we convert to \si{ft} because the rest of our quantitites are measured in feet too.

  It'll be easier in this case to consider the dimensions of $\mu$ before calculating it's magnitude:
  \begin{align*}
    [\mu] &= \frac{[\Delta y][\tau]}{[\Delta u]} = \frac{ \si{ft} \times \si{lb_f.ft^{-2}} }{ \si{ ft.s^{-1}} }
      = \si{ft.lb_f.ft^{-2}.ft^{-1}.s} = \si{lb_f.s.ft^{-2}}, \text{ so } \\
    \mu &= \frac{ 1/48 \times 7.7 }{25} \si{lb_f.s.ft^{-2}} = 6.42 \times 10^{-3} \si{lb_f.s.ft^{-2}}
  \end{align*}
  \begin{equation*}
    \boxed{\mu = 6.42 \times 10^{-3} \si{lb_f.s.ft^{-2}}}
  \end{equation*}
\end{solution}

%    ----    ----    Q 1.19    ----    ----

\begin{question}
  Q 1.19 A sample of motor oil has been tested in a flat plate shearing device with the following results: $\tau = 4.0 \si{lb_{f}/ft^2}$ for a plate separation distance of $0.05 \si{in}$ and a fluid viscosity of $\mu = 6.5 \times 10^{-3} \si{lb_f.s.ft^{-2}}$. Determine the top plate velocity.
\end{question}

\begin{solution} This question is similar to Q1.18.

  \uheading{Data given:}
  \begin{itemize}
    \item $\tau = 4.0 \si{lb_{f}.ft^{-2}}$,
    \item plate separation, $\Delta y = 0.05 \si{in} = 0.05 \x 1/12 \si{ft}$,
    \item fluid viscosity, $\mu = 6.5 \times 10^{-3} \si{lb_f.s.ft^{-2}}$
  \end{itemize}

  We are looking for the top plate velocity after the shear stress is applied. In our flat plate shearing device, the top plate begins at rest, so we are just looking for the change in velocity, $\Delta u$.

  \uheading{Relevant equation:}
  Newton's Law of Viscosity,
  \begin{equation}
    \tau = \mu \frac{ \Delta u }{\Delta y} \Rightarrow \Delta u = \frac{ \Delta y \tau}{\mu}
  \end{equation}.

  Hence
  \begin{align*}
    \Delta u &= \frac{ 0.5 \x 1/12 \si{ft} \x 4.0 \si{lb_f.ft^{-2}} }{ 6.5 \x 10^{-3} \si{lb_f.s.ft{^-2}} } \\
    \Delta u &= \frac{ 0.5 \x 1/12 \x 4.0 }{ 6.5 \x 10^{-3}} \si{ft.s^{-1}} = 25.64 \si{ft.s^{-1}}
  \end{align*}

  \begin{equation*}
    \boxed{ u_{\text{top plate}} =  25.64 \si{ft.s^{-1}}}
  \end{equation*}
\end{solution}

~\\\noindent \textbf{Note}: From here I have only done the even questions.

%    ----    ----    Q 1.20    ----    ----

\begin{question}
  Q 1.20 Suppose a water puddle is initially $1.5\si{mm}$ thick, and that the viscosity of water is $0.001 \si{kg.m^{-1}s^{-1}}$. If your hand is applying a shear stress of magnitude $0.10 \si{kg.m^{-1}.s^{-2}}$, calculate the shear strain rate, $d\gamma/dt$, in the fluid, the velocity gradient, $du/dy$, in the fluid, and the speed $u$ at which your hand is moving.
\end{question}

\begin{solution}

  \uheading{Data given:}
  \begin{itemize}
    \item thickness of sample (plate separation), $\Delta y = 1.5 \si{mm} = 1.5 \x 10^{-3} \si{m}$,
    \item viscosity of water, $\mu = 0.001 \si{kg.m^{-1}s^{-1}} = 1 \x 10^{-3} \si{kg.m^{-1}s^{-1}}$,
    \item shear stress applied, $\tau = 0.10 \si{kg.m^{-1}.s^{-2}} = 1 \x 10^{-1} \si{kg.m^{-1}.s^{-2}}$.
  \end{itemize}

  \uheading{Relevant equation:} Newton's Law of Viscosity.
  \begin{equation}
    \tau = \mu \frac{d \gamma}{dt} = \mu \frac{du}{dt} ~\Rightarrow~ \frac{d \gamma}{dt} = \frac{du}{dt} = \tau / \mu
  \end{equation}

  So we have
  \begin{align*}
    \frac{d \gamma}{dt} = \frac{du}{dy} = \frac{ 1 \x 10^{-1} \si{kg.m^{-1}.s^{-2}} }{ 1 \x 10^{-3} \si{kg.m^{-1}s^{-1}} } = 1 \x 10^{2} \si{s^{-1}}.
  \end{align*}

  We can now use this value for the velocity gradient to calculate the speed of the hand. Since we assume that the initial velocity of the surface of the puddle is zero, this will simply be the change in velocity, $d u$ (or $\Delta u$ -- this is just notation).

  \begin{align*}
    \frac{du}{dy} &= 1 \x 10^{2} \si{s^{-1}} \\
    \Rightarrow~ du &= 1 \x 10^{2} \si{s^{-1}} \x dy = 1 \x 10^{2} \si{s^{-1}} \x 1.5 \x 10^{-3} \si{m} \\
    du &= 0.15 \si{m.s^{-1}}
  \end{align*}

  \begin{equation*}
    \boxed{\frac{d \gamma}{dt} = \frac{du}{dy} = 0.01 \si{s^{-1}} \text{ and } u_{\text{hand}} = 0.15 \si{m.s^{-1}} }
  \end{equation*}
\end{solution}

%    ----    ----    Q 1.21    ----    ----
%    ----    ----    Q 1.22    ----    ----

\begin{question}
  Q 1.22 Consider the laminar flow of fluid between parallel flat plates. The velocity distribution for this flow is $u/u_{max} = 1 - (y/h)^2$. The fluid is water with viscosity $0.001 \si{kg.m^{-1}s^{-1}}$, $u_{max} = 0.4 \si{m.s^{-1}}$ and $h = 2.5 \si{mn}$.
\end{question}

%    ----    ----    Q 1.23    ----    ----
%    ----    ----    Q 1.24    ----    ----

%    ----    ----    Q 1.25    ----    ----
%    ----    ----    Q 1.26    ----    ----

%    ----    ----    Q 1.27    ----    ----
%    ----    ----    Q 1.28    ----    ----

%    ----    ----    Q 1.29    ----    ----
%    ----    ----    Q 1.30    ----    ----

%    ----    ----    Q 1.31    ----    ----
%    ----    ----    Q 1.32    ----    ----

%    ----    ----    Q 1.33    ----    ----
%    ----    ----    Q 1.34    ----    ----

%    ----    ----    Q 1.35    ----    ----
%    ----    ----    Q 1.36    ----    ----

%    ----    ----    Q 1.37    ----    ----
%    ----    ----    Q 1.38    ----    ----

%    ----    ----    Q 1.39    ----    ----
%    ----    ----    Q 1.40    ----    ----

%    ----    ----    Q 1.41    ----    ----
%    ----    ----    Q 1.42    ----    ----

%    ----    ----    Q 1.43    ----    ----
%    ----    ----    Q 1.44    ----    ----

%    ----    ----    Q 1.45    ----    ----
%    ----    ----    Q 1.46    ----    ----

%    ----    ----    Q 1.47    ----    ----
%    ----    ----    Q 1.48    ----    ----

%    ----    ----    Q 1.49    ----    ----
%    ----    ----    Q 1.50    ----    ----

%    ----    ----    Q 1.51    ----    ----
%    ----    ----    Q 1.52    ----    ----

%    ----    ----    Q 1.53    ----    ----
%    ----    ----    Q 1.54    ----    ----

%    ----    ----    Q 1.55    ----    ----
%    ----    ----    Q 1.56    ----    ----

%    ----    ----    Q 1.57    ----    ----
%    ----    ----    Q 1.58    ----    ----

%    ----    ----    Q 1.59    ----    ----
%    ----    ----    Q 1.60    ----    ----
