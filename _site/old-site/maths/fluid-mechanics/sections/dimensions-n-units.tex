\section{An important aside on dimensions and units}

But what is the difference between a dimension and a unit?

\begin{definition}
  A dimension is a physical variable used to specific some charactersitics of a system. For example, length, mass and temperature are all dimensions.
\end{definition}

\begin{definition}
  A unit is a particular amount of a physical quantity or dimension. For example, leagues is a (pretty bad) unit of length, micrograms is a unit of mass and degrees Centigrade is a unit of temperature.
\end{definition}

The dimensions of length $[L]$, time $[t]$ and temperature $[T]$ are considered to be base dimensions and are included in all dimension sets. A dimension set of the form $[MLtT]$ or $[FLtT]$ describe most problems in fluid mechanics. Other fields might require more dimensions, for example charge.

What unit set we choose will dictate how we understand the laws of physics. Consider
\begin{equation}
  \vec{F} \propto M \vec{a}
\end{equation}
in the usual notation. The proportionality symbol is required because there is no a priori reason to assume that the units of measure will automatically be consistent with $\vec{F} = M \vec{a}$. More generally we write
\begin{equation}
  \vec{F} = \frac{M \vec{a}}{g_c},
\end{equation}
where $g_c$ will be determined by our choice of units. However the underlying dimensisonal relationship \textit{is} true a priori:
\begin{equation}
  [F] = [M][Lt^{-2}].
\end{equation}

\subsection*{Système International d'Unités (SI)}

This is an $[MLtT]$ system, meaning units for mass, length, time and temperature are all indepedentenly defined:
\begin{itemize}
  \item mass: kilogram (kg),
  \item length: metre (m),
  \item time: second (s),
  \item temperature: Kelvin (K).
\end{itemize}

The SI system also has a composite unit for force, the Newton:

\begin{equation*}
  1 \si{.N} = 1 \si{.kg.m.s^{-2}}.
\end{equation*}

The SI system has ben defined so that $g_c$ and many other proportionality constants (including Fourier's Law and Gauss' Law) are dimensionless constants of unit magnitude. You are welcome.

\subsection*{Centimetre-gram-second system (cgs)}

This is a ripoff of the SI system but for smaller stuff which wouldn't make sense to measure in such large units as the metre and the kilogram:
\begin{itemize}
  \item mass: gram (g),
  \item length: centimetre (cm),
  \item time: second (s),
  \item temperature: Kelvin (K).
\end{itemize}

\subsection*{British Gravitational Unit System (BG)}

This is a $[FLtT]$ system:
\begin{itemize}
  \item force: pound force (\si{lb_f}),
  \item length: foot (ft),
  \item time: second (s),
  \item temperature: degree Rakine (ºR).
\end{itemize}

The BG system also has a composite unit, but this time for mass, which is defined in such a way that allows for $g_c=1$:

\begin{equation*}
  1 \text{ slug} = 1 \si{lb_f.s^2.ft^{-1}},
\end{equation*}

and the weight of 1 slug is 32.2 \si{lb_f}, which is in no way a silly thing to name a unit.

\subsection*{English Engineering Unit System (EE)}

This is an $[FMLtT]$ system and where good, sane, ways of measuring things to die:
\begin{itemize}
  \item force: pound force (\si{lb_f}),
  \item force: pound mass (\si{lb_m}),
  \item length: foot (ft),
  \item time: second (s),
  \item temperature: degree Rakine (ºR).
\end{itemize}

The force unit is of course arbitrarily defined so that 1 \si{lb_m} weights 1 \si{lb_f} when acted upon by gravity at sea level. In this case we have $g_c = 32.2 $. Fun fact: none of this makes any sense.
