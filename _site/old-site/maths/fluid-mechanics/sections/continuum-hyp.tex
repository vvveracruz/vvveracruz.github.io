\section{The Continuum Hypothesis}

\begin{proposition}[The Continuum Hypothesis]
  We assume that a fluid can be treated as a continuous substance rather than a group of discrete molecules.
\end{proposition}

Calculating the motion of all the molecules in a fluid body is prohibitively expensive and unnecessary for a wide range of applications where models based on the continuum hypothesis describe reality well. On key benefit of the continuum hypothesis is that it allows dor the use of differential calculus to find solutions to the models of fluid mechanics since we consider each fluid property to be a continuous function of position and time.

It is important to note that this model represents reality \textit{well enough} but it is not reality itself. We are making assumptions about fluid behaviour that we know break down for sufficiently small quantities, specifically when the length scale of a physical phenomenon is of the same order as molecular dimensions of practical importance. An important quanitity of this kind is the the average distance travelled by a molecule between successive impact, known as the \textit{mean free path}. This is an important consideration when applying the continuum hypothesis to gases. In fluids, the mean distance between molecules is very small, so our theory has a wider range of applications.

It is always a good idea to check all the assumptions we are making before solving a problem using any given method.

\subsection*{The Lagrangian Description}

The Lagrangian Description is a continuum description (so we can use calculus). The key idea here is that we consider a conceptual fluid particle -- a shapeless blob of fluid which contains mass. We then assume our fluid is a collection of such discrete fluid particles. We must also assume these fluid particles are small enough to allow for us to treat the fluid as if it was infinitely divisible. Crucially, these are \textbf{not} molecules. We are still considering the fluid to be a continuum, except now we are cutting this continuum up into smaller fluid particles.

A classic example of the Lagrangian approach is dropping droplets of dye into a moving body of water. You would be able to follow the path of the droplet of dye as it moves within the fluid: this is a fluid particle.

This desciption is conceptually similar to the idea of treating bodies as point masses in classical mechanics.

\subsection*{The Eulerian Description}

The Eulerian description is a different kind of continuum description. In this case, instead of looking at a particular fluid particle and watching how its position changes within the fluid, we fix the position and look at how the velocity field changes with time. Fluid motion in the Eulerian description has fluid properties applied to specific places in space, instead of particular fluid particles.

The Eulerian approach constitutes a field-like description of fluid motion much like we see fields electromagnetism or classical understanding of Gravity. This is particularly useful because fields are well understood mathematically. We can easily apply vector calculus to problems formulated using the Eulerian Description to find solutions to our problems. 
